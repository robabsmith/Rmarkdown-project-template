\documentclass[10pt,twoside]{article}
\usepackage{lmodern}
\usepackage{amssymb,amsmath}
\usepackage{ifxetex,ifluatex}
\usepackage{fixltx2e} % provides \textsubscript
\ifnum 0\ifxetex 1\fi\ifluatex 1\fi=0 % if pdftex
  \usepackage[T1]{fontenc}
  \usepackage[utf8]{inputenc}
\else % if luatex or xelatex
  \ifxetex
    \usepackage{mathspec}
  \else
    \usepackage{fontspec}
  \fi
  \defaultfontfeatures{Ligatures=TeX,Scale=MatchLowercase}
\fi
% use upquote if available, for straight quotes in verbatim environments
\IfFileExists{upquote.sty}{\usepackage{upquote}}{}
% use microtype if available
\IfFileExists{microtype.sty}{%
\usepackage{microtype}
\UseMicrotypeSet[protrusion]{basicmath} % disable protrusion for tt fonts
}{}
\usepackage[top = 25mm, left=25mm, right=25mm, bottom=25mm, paperwidth=170mm,
paperheight=240mm]{geometry}
\usepackage{hyperref}
\PassOptionsToPackage{usenames,dvipsnames}{color} % color is loaded by hyperref
\hypersetup{unicode=true,
            colorlinks=true,
            linkcolor=red,
            citecolor=Blue,
            urlcolor=black,
            breaklinks=true}
\urlstyle{same}  % don't use monospace font for urls
\usepackage{graphicx,grffile}
\makeatletter
\def\maxwidth{\ifdim\Gin@nat@width>\linewidth\linewidth\else\Gin@nat@width\fi}
\def\maxheight{\ifdim\Gin@nat@height>\textheight\textheight\else\Gin@nat@height\fi}
\makeatother
% Scale images if necessary, so that they will not overflow the page
% margins by default, and it is still possible to overwrite the defaults
% using explicit options in \includegraphics[width, height, ...]{}
\setkeys{Gin}{width=\maxwidth,height=\maxheight,keepaspectratio}
\IfFileExists{parskip.sty}{%
\usepackage{parskip}
}{% else
\setlength{\parindent}{0pt}
\setlength{\parskip}{6pt plus 2pt minus 1pt}
}
\setlength{\emergencystretch}{3em}  % prevent overfull lines
\providecommand{\tightlist}{%
  \setlength{\itemsep}{0pt}\setlength{\parskip}{0pt}}
\setcounter{secnumdepth}{5}
% Redefines (sub)paragraphs to behave more like sections
\ifx\paragraph\undefined\else
\let\oldparagraph\paragraph
\renewcommand{\paragraph}[1]{\oldparagraph{#1}\mbox{}}
\fi
\ifx\subparagraph\undefined\else
\let\oldsubparagraph\subparagraph
\renewcommand{\subparagraph}[1]{\oldsubparagraph{#1}\mbox{}}
\fi

%%% Use protect on footnotes to avoid problems with footnotes in titles
\let\rmarkdownfootnote\footnote%
\def\footnote{\protect\rmarkdownfootnote}

%%% Change title format to be more compact
\usepackage{titling}

% Create subtitle command for use in maketitle
\providecommand{\subtitle}[1]{
  \posttitle{
    \begin{center}\large#1\end{center}
    }
}

\setlength{\droptitle}{-2em}

  \title{}
    \pretitle{\vspace{\droptitle}}
  \posttitle{}
    \author{}
    \preauthor{}\postauthor{}
    \date{}
    \predate{}\postdate{}
  
\usepackage{booktabs}
\usepackage{longtable}
\usepackage{array}
\usepackage{multirow}
\usepackage{wrapfig}
\usepackage{float}
\usepackage{colortbl}
\usepackage{pdflscape}
\usepackage{tabu}
\usepackage{threeparttable}
\usepackage{threeparttablex}
\usepackage[normalem]{ulem}
\usepackage{makecell}
\usepackage{xcolor}

\usepackage{placeins}
\usepackage{fancyhdr}
\usepackage{setspace}
\onehalfspacing
\usepackage{chngcntr}
\counterwithin{figure}{section}
\counterwithin{table}{section}
\counterwithin{equation}{section}
\counterwithin{footnote}{section}
\usepackage{subfig}
\usepackage{float}
\usepackage{lscape}
\newcommand{\blandscape}{\begin{landscape}}
\newcommand{\elandscape}{\end{landscape}}
\renewcommand{\thepage}{(\thesection):\arabic{page}}
\newcommand{\onlythepage}{\arabic{page}}
\newcommand*{\secref}[1]{Section~\ref{#1}}
\raggedbottom

\begin{document}

\pagenumbering{gobble}

\begin{centering}

\vspace{2 cm}

\Large

{\bf Title of your project}

\vspace{2 cm}

\Large
Your name(s) 1\\
Your name(s) 2\\
Your name(s) 3

\vspace{2 cm}

\normalsize
Supervisor: Supervisor's name

\vspace{2 cm}

\normalsize
Submitted in partial fulfilment for the fourth semester project

Spring 2020

\vspace{2 cm}

\normalsize
The Department of Business and Management

\normalsize
Aalborg University

\end{centering}

\newpage

\pagestyle{fancy}

\fancyhead[LE,RO]{}
\fancyhead[LO,RE]{}

\renewcommand{\headrulewidth}{0.4pt}
\renewcommand{\footrulewidth}{0pt}

\pagenumbering{roman}

\FloatBarrier
\newpage

\fancyhead[CO,CE]{Acknowledgements}
\section*{Acknowledgements}
\addcontentsline{toc}{section}{Acknowledgements}

I just want to thank my chicken for not eating my corn while i wrote
this project, and my dog for not eating the project when I was done.

\FloatBarrier
\cleardoublepage

\fancyhead[CO,CE]{Table of Contents}

\setcounter{tocdepth}{2}

\tableofcontents

\cleardoublepage
\FloatBarrier

\fancyhead[CO,CE]{Part 2 header for page - change this}

\pagenumbering{arabic}

\begin{centering}

\vspace{1 cm}

\Huge

\section*{Header for section}
\addcontentsline{toc}{section}{Header you want in the TOC for part 1}
\addtocounter{section}{1}

\vspace{1 cm}

\Large
Name 1

\normalsize
Supervisor: Supervisor name

\vspace{1 cm}

\Large


\vspace{1 cm}

\normalsize
The Department of Business and Management

Aalborg University

\vspace{1 cm}

\end{centering}

\FloatBarrier
\newpage

\fancyhead[CO,CE]{Part 2 header for page - change this, \arabic{page}}

\hypertarget{first-section-not-numbered-because-of-the-curly-bracket-dash--}{%
\subsection*{First section, not numbered because of the curly bracket
dash
\{-\}}\label{first-section-not-numbered-because-of-the-curly-bracket-dash--}}
\addcontentsline{toc}{subsection}{First section, not numbered because of
the curly bracket dash \{-\}}

Curabitur pretium tincidunt lacus. Nulla gravida orci a odio. Nullam
varius, turpis et commodo pharetra, est eros bibendum elit, nec luctus
magna felis sollicitudin mauris. Brown \& Graf (2013) noted that Integer
in mauris eu nibh euismod gravida. Duis ac tellus et risus vulputate
vehicula. Donec lobortis risus a elit. Etiam Lorem ipsum dolor sit amet,
consectetur adipiscing elit, sed do eiusmod tempor incididunt ut labore
et dolore magna aliqua. Ut enim ad minim veniam, quis nostrud
exercitation ullamco laboris nisi ut aliquip ex ea commodo consequat.
Duis aute irure dolor in reprehenderit in voluptate velit esse cillum
dolore eu fugiat nulla pariatur. Excepteur sint occaecat cupidatat non
proident, sunt in culpa qui officia deserunt mollit anim id est laborum.

Andersen, Duus, \& Jensen (2016, p. 25)

Justiniano, Primiceri, \& Tambalotti (2015, pp. 10--25)

(See, Zinman, 2015, p. 15)

Curabitur pretium tincidunt lacus. Nulla gravida orci a odio. Nullam
varius, turpis et commodo pharetra, est eros bibendum elit, nec luctus
magna felis sollicitudin mauris. Integer in mauris, as suggested by
Walks (2013, pp. 10--25), eu nibh euismod gravida. Duis ac tellus et
risus vulputate vehicula. Donec lobortis risus a elit. Etiam tempor. Ut
ullamcorper, ligula eu tempor congue, eros est euismod turpis, id
tincidunt sapien risus a quam. Maecenas fermentum consequat mi. Donec
fermentum. Pellentesque malesuada nulla a mi. Duis sapien sem, aliquet
nec, commodo eget, consequat quis, neque. Aliquam faucibus, elit ut
dictum aliquet, felis nisl adipiscing sapien, sed malesuada diam lacus
eget erat. Cras mollis scelerisque nunc. Nullam arcu. Aliquam consequat.
Curabitur augue lorem, dapibus quis, laoreet et, pretium ac, nisi.
Aenean magna nisl, mollis quis, molestie eu, feugiat in, orci. In hac
habitasse platea dictumst.

\hypertarget{sec:part1-sec2}{%
\subsection{Second section numbered}\label{sec:part1-sec2}}

Curabitur pretium tincidunt lacus. Nulla gravida orci a odio. Nullam
varius, turpis et commodo pharetra, est eros bibendum elit, nec luctus
magna felis sollicitudin mauris. Integer in mauris eu nibh euismod
gravida. (See, Zinman, 2015, p. 15) Duis ac tellus et risus vulputate
vehicula. Donec lobortis risus a elit. Etiam Lorem ipsum dolor sit amet,
consectetur adipiscing elit, sed do eiusmod tempor incididunt ut labore
et dolore magna aliqua. Ut enim ad minim veniam, quis nostrud
exercitation ullamco laboris nisi ut aliquip ex ea commodo consequat.
Duis aute irure dolor in reprehenderit in voluptate velit esse cillum
dolore eu fugiat nulla pariatur. Excepteur sint occaecat cupidatat non
proident, sunt in culpa qui officia deserunt mollit anim id est laborum.

\FloatBarrier
\cleardoublepage

\fancyhead[CO,CE]{Part 2 header for page - change this}

\begin{centering}

\vspace{1 cm}

\Huge

\section*{Header for section}
\addcontentsline{toc}{section}{Header you want in the TOC for part 1}
\addtocounter{section}{1}

\vspace{1 cm}

\Large
Name 1

\normalsize
Supervisor: Supervisor name

\vspace{1 cm}

\Large


\vspace{1 cm}

\normalsize
The Department of Business and Management

Aalborg University

\vspace{1 cm}

\end{centering}

\FloatBarrier
\newpage

\fancyhead[CO,CE]{Literature Review, \arabic{page}}

\hypertarget{sec:part2-sec1}{%
\subsection{First section}\label{sec:part2-sec1}}

Curabitur pretium tincidunt lacus. Nulla gravida orci a odio. Nullam
varius, turpis et commodo pharetra, est eros bibendum elit, nec luctus
magna felis sollicitudin mauris. Integer in mauris eu nibh euismod
gravida. Duis ac tellus et risus vulputate vehicula. Donec lobortis
risus a elit. Etiam Lorem ipsum dolor sit amet, consectetur adipiscing
elit, sed do eiusmod tempor incididunt ut labore et dolore magna aliqua.
Ut enim ad minim veniam, quis nostrud exercitation ullamco laboris nisi
ut aliquip ex ea commodo consequat. Duis aute irure dolor in
reprehenderit in voluptate velit esse cillum dolore eu fugiat nulla
pariatur. Excepteur sint occaecat cupidatat non proident, sunt in culpa
qui officia deserunt mollit anim id est laborum.

\emph{Some random italics text}

\textbf{Some random bold text}

A pretty cool reference back to the first section can be made
automatically like this: \ref{sec:part1-sec2}

The standard command will just generate a number, like this :
\ref{sec:part1-sec2}

You can also use the command created in the ``ProjectBuilder'' to
customise your references.

This command includes the word ``Section'' before the reference:
Section~\ref{sec:part1-sec2}

You can do the same for figures and tables. If you label them correctly,
then they will automatically be added to the list of tables and figures
at the end of the document.

Curabitur pretium tincidunt lacus. \emph{Some random italics text}, est
eros bibendum elit, nec luctus magna felis sollicitudin mauris. Integer
in mauris eu nibh euismod gravida. Duis ac tellus et risus vulputate
vehicula. Donec lobortis risus a elit. Etiam tempor. Ut ullamcorper,
ligula eu tempor congue, eros est euismod turpis, id tincidunt sapien
risus a quam. Maecenas fermentum consequat mi. Donec fermentum.
Pellentesque malesuada nulla a mi. Duis sapien sem, aliquet nec, commodo
eget, consequat quis, neque. Aliquam faucibus, elit ut dictum aliquet,
felis nisl adipiscing sapien, sed malesuada diam lacus eget erat. Cras
mollis scelerisque nunc. Nullam arcu. Aliquam consequat. Curabitur augue
lorem, dapibus quis, laoreet et, pretium ac, nisi. Aenean magna nisl,
mollis quis, molestie eu, feugiat in, orci. In hac habitasse platea
dictumst.

\hypertarget{sec:part2-sec2}{%
\subsection{Second section numbered}\label{sec:part2-sec2}}

Curabitur pretium tincidunt lacus. Nulla gravida orci a odio. Nullam
varius, turpis et commodo pharetra, est eros bibendum elit, nec luctus
magna felis sollicitudin mauris. Integer in mauris eu nibh euismod
gravida. Duis ac tellus et risus vulputate vehicula. Donec lobortis
risus a elit. Etiam Lorem ipsum dolor sit amet, consectetur adipiscing
elit, sed do eiusmod tempor incididunt ut labore et dolore magna aliqua.
Ut enim ad minim veniam, quis nostrud exercitation ullamco laboris nisi
ut aliquip ex ea commodo consequat. Duis aute irure dolor in
reprehenderit in voluptate velit esse cillum dolore eu fugiat nulla
pariatur. Excepteur sint occaecat cupidatat non proident, sunt in culpa
qui officia deserunt mollit anim id est laborum.

\hypertarget{sec:part2-sec2-subsec1}{%
\subsubsection{A third level heading}\label{sec:part2-sec2-subsec1}}

Integer in mauris eu nibh euismod gravida. Duis ac tellus et risus
vulputate vehicula. Donec lobortis risus a elit. Etiam Lorem ipsum dolor
sit amet, consectetur adipiscing elit, sed do eiusmod tempor incididunt
ut labore et dolore magna aliqua. Ut enim ad minim veniam, quis nostrud
exercitation ullamco laboris nisi ut aliquip ex ea commodo consequat.
Duis aute irure dolor in reprehenderit in voluptate velit esse cillum
dolore eu fugiat nulla pariatur. Excepteur sint occaecat cupidatat non
proident, sunt in culpa qui officia deserunt mollit anim id est laborum.

\begin{quote}
"quoted text of some fancy academic somewhere\ldots{}Duis ac tellus et
risus vulputate vehicula. Donec lobortis risus a elit. Etiam tempor. Ut
ullamcorper, ligula eu tempor congue, eros est euismod turpis, id
tincidunt sapien risus a quam. Maecenas fermentum consequat mi. Donec
fermentum. Pellentesque malesuada nulla a mi. Duis sapien sem, aliquet
nec, commodo eget, consequat quis, neque. Aliquam faucibus, elit ut
dictum aliquet, felis nisl adipiscing sapien, sed malesuada diam lacus
eget erat. Cras mollis scelerisque nunc. Nullam arcu. Aliquam consequat.
Curabitur augue lorem, dapibus quis, laoreet et, pretium ac, nisi.
Aenean magna nisl, mollis quis, molestie eu, feugiat in, orci. In hac
habitasse platea dictumst.
\end{quote}

\hypertarget{sec:part2-sec2-subsec2}{%
\subsubsection{A second level 3 heading}\label{sec:part2-sec2-subsec2}}

This section contains some fancy equations these can be entered in a
couple of ways, but the LaTeX version is probably the most adaptable and
consitent. in text you can just type something in single \$ marks, and
it will stay in line,
\(y = \beta_0 + \beta _1 ln(x^2) + \frac{ac}{x^2_1}\).

Or you can type it in double \$\$ marks and it will float the equation
in the middle of the page with white space around it.

\[y = \beta_0 + \beta _1 ln(x^2) + \frac{ac}{x^2_1}\]

If you want to get really fancy and to be able to reference your
equations automatically, you can also add a label, but this is easier to
do in the LaTeX version.

\begin{equation}
y = \beta_0 + \beta _1 ln(x^2) + \frac{ac}{x^2_1}
\label{eq:fancy-equation}
\end{equation}

This version allows you to do a reference to Equation
(\ref{eq:fancy-equation}) in your text. If you look at the code, you
will see that there are no section numbers or equation numbers ``hard
coded'' into the text. If I add another equation above this one, it will
just adust the numbering automatically.

The strucutre of the numbering of different elements is set in the
\texttt{YAML} of the ``ProjectBuilder''. There you will find code that
looks like this
\texttt{\textbackslash{}counterwithin\{table\}\{section\}} - this means,
that every time a level 1 section is started, the counter for tables
will start from 1 again.

\hypertarget{for-inserting-images}{%
\subsection{For inserting images}\label{for-inserting-images}}

\FloatBarrier
\cleardoublepage

\pagenumbering{roman}

\fancyhead[LO,RE]{}
\fancyhead[CO,CE]{List of Figures, \arabic{page}}

\listoffigures
\addcontentsline{toc}{section}{List of Figures}

\FloatBarrier
\cleardoublepage

\fancyhead[CO,CE]{List of Tables, \arabic{page}}

\listoftables
\addcontentsline{toc}{section}{List of Tables}

\newpage
\FloatBarrier

\cleardoublepage
\fancyhead[CO,CE]{References, \arabic{page}}

\hypertarget{references}{%
\section*{References}\label{references}}
\addcontentsline{toc}{section}{References}

\hypertarget{refs}{}
\leavevmode\hypertarget{ref-Andersen2016}{}%
Andersen, A. L., Duus, C., \& Jensen, T. L. (2016). Household debt and
spending during the financial crisis: Evidence from Danish micro data.
\emph{European Economic Review}, \emph{89}, 96--115. Retrieved from
\url{http://linkinghub.elsevier.com/retrieve/pii/S0014292116301106}

\leavevmode\hypertarget{ref-Brown2013}{}%
Brown, M., \& Graf, R. (2013). Financial Literacy, Household Investment
and Household Debt: Evidence from Switzerland. \emph{Working Papers on
Finance}. Retrieved from
\url{http://www1.vwa.unisg.ch/RePEc/usg/sfwpfi/WPF-1301.pdf}

\leavevmode\hypertarget{ref-Justiniano2015b}{}%
Justiniano, A., Primiceri, G., \& Tambalotti, A. (2015). Household
leveraging and deleveraging. \emph{Review of Economic Dynamics},
\emph{18}(1), 3--20.

\leavevmode\hypertarget{ref-Walks2013}{}%
Walks, A. (2013). Mapping the Urban Debtscape: The Geography of
Household Debt in Canadian Cities. \emph{Urban Geography}, \emph{34}(2),
153--187. Routledge.

\leavevmode\hypertarget{ref-Zinman2015}{}%
Zinman, J. (2015). Household Debt: Facts, Puzzles, Theories, and
Policies. \emph{Annual Review of Economics}, \emph{7}(1), 251--276.
Annual Reviews. Retrieved from
\url{http://www.annualreviews.org/doi/10.1146/annurev-economics-080614-115640}


\end{document}
