\documentclass[10pt,twoside]{article}
\usepackage{lmodern}
\usepackage{amssymb,amsmath}
\usepackage{ifxetex,ifluatex}
\usepackage{fixltx2e} % provides \textsubscript
\ifnum 0\ifxetex 1\fi\ifluatex 1\fi=0 % if pdftex
  \usepackage[T1]{fontenc}
  \usepackage[utf8]{inputenc}
\else % if luatex or xelatex
  \ifxetex
    \usepackage{mathspec}
  \else
    \usepackage{fontspec}
  \fi
  \defaultfontfeatures{Ligatures=TeX,Scale=MatchLowercase}
\fi
% use upquote if available, for straight quotes in verbatim environments
\IfFileExists{upquote.sty}{\usepackage{upquote}}{}
% use microtype if available
\IfFileExists{microtype.sty}{%
\usepackage{microtype}
\UseMicrotypeSet[protrusion]{basicmath} % disable protrusion for tt fonts
}{}
\usepackage[top = 25mm, left=25mm, right=25mm, bottom=25mm, paperwidth=170mm,
paperheight=240mm]{geometry}
\usepackage{hyperref}
\PassOptionsToPackage{usenames,dvipsnames}{color} % color is loaded by hyperref
\hypersetup{unicode=true,
            colorlinks=true,
            linkcolor=red,
            citecolor=Blue,
            urlcolor=black,
            breaklinks=true}
\urlstyle{same}  % don't use monospace font for urls
\usepackage{graphicx,grffile}
\makeatletter
\def\maxwidth{\ifdim\Gin@nat@width>\linewidth\linewidth\else\Gin@nat@width\fi}
\def\maxheight{\ifdim\Gin@nat@height>\textheight\textheight\else\Gin@nat@height\fi}
\makeatother
% Scale images if necessary, so that they will not overflow the page
% margins by default, and it is still possible to overwrite the defaults
% using explicit options in \includegraphics[width, height, ...]{}
\setkeys{Gin}{width=\maxwidth,height=\maxheight,keepaspectratio}
\IfFileExists{parskip.sty}{%
\usepackage{parskip}
}{% else
\setlength{\parindent}{0pt}
\setlength{\parskip}{6pt plus 2pt minus 1pt}
}
\setlength{\emergencystretch}{3em}  % prevent overfull lines
\providecommand{\tightlist}{%
  \setlength{\itemsep}{0pt}\setlength{\parskip}{0pt}}
\setcounter{secnumdepth}{5}
% Redefines (sub)paragraphs to behave more like sections
\ifx\paragraph\undefined\else
\let\oldparagraph\paragraph
\renewcommand{\paragraph}[1]{\oldparagraph{#1}\mbox{}}
\fi
\ifx\subparagraph\undefined\else
\let\oldsubparagraph\subparagraph
\renewcommand{\subparagraph}[1]{\oldsubparagraph{#1}\mbox{}}
\fi

%%% Use protect on footnotes to avoid problems with footnotes in titles
\let\rmarkdownfootnote\footnote%
\def\footnote{\protect\rmarkdownfootnote}

%%% Change title format to be more compact
\usepackage{titling}

% Create subtitle command for use in maketitle
\providecommand{\subtitle}[1]{
  \posttitle{
    \begin{center}\large#1\end{center}
    }
}

\setlength{\droptitle}{-2em}

  \title{}
    \pretitle{\vspace{\droptitle}}
  \posttitle{}
    \author{}
    \preauthor{}\postauthor{}
    \date{}
    \predate{}\postdate{}
  
\usepackage{booktabs}
\usepackage{longtable}
\usepackage{array}
\usepackage{multirow}
\usepackage{wrapfig}
\usepackage{float}
\usepackage{colortbl}
\usepackage{pdflscape}
\usepackage{tabu}
\usepackage{threeparttable}
\usepackage{threeparttablex}
\usepackage[normalem]{ulem}
\usepackage{makecell}
\usepackage{xcolor}

\usepackage{placeins}
\usepackage{fancyhdr}
\usepackage{setspace}
\onehalfspacing
\usepackage{chngcntr}
\counterwithin{figure}{section}
\counterwithin{table}{section}
\counterwithin{equation}{section}
\counterwithin{footnote}{section}
\usepackage{subfig}
\usepackage{float}
\usepackage{lscape}
\newcommand{\blandscape}{\begin{landscape}}
\newcommand{\elandscape}{\end{landscape}}
\renewcommand{\thepage}{(\thesection):\arabic{page}}
\newcommand{\onlythepage}{\arabic{page}}
\newcommand*{\secref}[1]{Section~\ref{#1}}
\raggedbottom

\begin{document}

\pagenumbering{gobble}

\begin{centering}

\vspace{2 cm}

\Large

{\bf Title of your project}

\vspace{2 cm}

\Large
Your name(s) 1\\
Your name(s) 2\\
Your name(s) 3

\vspace{2 cm}

\normalsize
Supervisor: Supervisor's name

\vspace{2 cm}

\normalsize
Submitted in partial fulfilment for the fourth semester project

Spring 2020

\vspace{2 cm}

\normalsize
The Department of Business and Management

\normalsize
Aalborg University

\end{centering}

\newpage

\pagestyle{fancy}

\fancyhead[LE,RO]{}
\fancyhead[LO,RE]{}

\renewcommand{\headrulewidth}{0.4pt}
\renewcommand{\footrulewidth}{0pt}

\pagenumbering{roman}

\FloatBarrier
\newpage

\fancyhead[CO,CE]{Acknowledgements}
\section*{Acknowledgements}
\addcontentsline{toc}{section}{Acknowledgements}

I just want to thank my chicken for not eating my corn while i wrote
this project, and my dog for not eating the project when I was done.

\FloatBarrier
\cleardoublepage

\fancyhead[CO,CE]{Table of Contents}

\setcounter{tocdepth}{2}

\tableofcontents
\addcontentsline{toc}{section}{Table of contents}

\cleardoublepage
\FloatBarrier

\fancyhead[CO,CE]{Part 1 header for the titlepage and onwards}

\pagenumbering{arabic}

\begin{centering}

\vspace{1 cm}

\Huge

\section*{Header for section}
\addcontentsline{toc}{section}{Header you want in the TOC for part 1}
\addtocounter{section}{1}

\vspace{1 cm}

\Large
Name 1

\normalsize
Supervisor: Supervisor name

\vspace{1 cm}

\Large


\vspace{1 cm}

\normalsize
The Department of Business and Management

Aalborg University

\vspace{1 cm}

\end{centering}

\cleardoublepage
\FloatBarrier

\fancyhead[CO,CE]{Guide to supervision}

\pagenumbering{arabic}

\hypertarget{sec:intro-to-supervision}{%
\subsection{Introduction to
supervision}\label{sec:intro-to-supervision}}

This is an introduction text that outlines that I will be your
supervisor for your project for 4. semester.

As (I hope) you can see from this text, my first language is English. So
anyone that wants to work on their English writing, or just get more
exposure to economics in English it might be a good opportunity for you
-- If you want to work outside of the country or in most multinationals
here in DK it is typically a requirement (Danmarks Nationalbank
included). You will not be assessed on your grammar, but you will need
to make sense and write in a professional manner.

If you would prefer to work in Danish, you are of course welcome to.

To get the most out of the supervision I recommend that in addition to
the introductory meeting, 3 group meetings should be sufficient.

\begin{enumerate}
\def\labelenumi{\arabic{enumi}.}
\setcounter{enumi}{-1}
\item
  \textbf{Introductory metting:} We will go through literature search
  and a few useful tools for writing in a collaborative project.
\item
  \textbf{Meeting 1:} You need to have worked on and bring a complete
  problem statement (see the guide and tips below), we will discuss it
  in the first meeting.
\item
  \textbf{Meeting 2:} The literature review, and expected method should
  be done, and any data or materials you plan to use should be
  collected. We will go through your planned method and argumentation in
  the meeting.
\item
  \textbf{Meeting 3:} The analysis should be complete and you should
  have some working points for your discussion / conclusions. We will go
  through your arguments verbally, and I will probe any major gaps I see
  in your thinking.
\end{enumerate}

\hypertarget{guidelines-for-supervision}{%
\subsection{Guidelines for
supervision:}\label{guidelines-for-supervision}}

\begin{enumerate}
\def\labelenumi{\arabic{enumi}.}
\item
  Any team member can communicate with me via Teams on behalf of your
  group. I expect that all communication has been discussed an agreed
  upon.
\item
  Just as you can expect me to read and provide comments on the days of
  meetings, I expect you to respect the deadlines you choose.
\item
  If you want something read before the meeting, it must be sent to me
  at least 2 working days before the meeting, I.e. Midnight Thursday for
  a Monday meeting. (Max 10 pages per meeting)
\item
  I will read and comment generally on the work, but will not make
  decisions for you. Your ability to choose and apply the correct
  methods is part of what you will be assessed on.
\item
  Each meeting is planned for one hour. I recommend that all groups have
  their first meeting with me\textbf{before
  16\textsuperscript{th}March}.
\item
  For every meeting you should bring with:

  \begin{enumerate}
  \def\labelenumii{\alph{enumii}.}
  \item
    Your problem statement (as it evolves with your work).
  \item
    A list of literature that you have covered up to that point (only
    the literature you have already read).
  \item
    The date by which you will be ready for the next meeting.
  \end{enumerate}
\end{enumerate}

\hypertarget{examinations}{%
\subsection{Examinations:}\label{examinations}}

You can write and be examined in Danish or English. If you choose
Danish, it might be the case that one of our Danish speaking staff will
join in the examination, 1x external examiner + me + possibly 1x Danish
AAU examiner. This will depend on Departmental resources, but you will
not be disadvantaged in any way because of any limitations that I might
have with the Danish language.

\hypertarget{leave-periods-absenteeism}{%
\subsection{Leave periods
(absenteeism):}\label{leave-periods-absenteeism}}

I will be away from Aalborg for the following periods:

\begin{enumerate}
\def\labelenumi{\arabic{enumi}.}
\item
  26\textsuperscript{th}March 2020 to the 29\textsuperscript{th} March
  2020. (Copenhagen)
\item
  04\textsuperscript{th} April 2020 to the 17\textsuperscript{th} April
  2020. (South Africa)
\end{enumerate}

\hypertarget{rough-guide-to-project-structures}{%
\subsection{Rough guide to project
structures}\label{rough-guide-to-project-structures}}

This is a\textbf{very rough}guide to writing a project. It is intended
to give you a very basic idea of what to include in a good project.

In terms of pages, each group will know how many people they have, the
official \textbf{maximum number of pages} (by character count, 2400
key-strokes including spaces) are:

1 person: 15 pages

2 People: 25 pages

3 people: 35 pages

4 people: 40pages

\textbf{Filling the pages is not the goal}, and you will not be given a
higher grade for filling all of your allocated pages with pointless
text. You will also not be penalised if you can get your message across
clearly in fewer pages. Keep in mind, that the average journal article
is roughly 15 -- 25 double spaced pages (around 8000 words).

You only need to address \textbf{one} problem, and to do it as well as
possible.

The written project is intended\textbf{to communicate}that you have done
your homework on your subject. This means that as a student you should
be able to demonstrate that you:

\begin{enumerate}
\def\labelenumi{\arabic{enumi}.}
\item
  Can identify an economic problem (or gap in the literature) that you
  think needs to be addressed (and why?!).
\item
  Can find, read and understand literature about the problem, and how
  others have dealt with it (reading and organising literature).
\item
  Can find the relevant information or data that you need to assess the
  problem, and that you know what to do with it when you do find it
  (number 2 helps with this) (data and methods).
\item
  Can present your findings in a well written document, where you give
  credit to all the authors that helped you to understand the problem
  (references).
\item
  If you make a statement, you either need to back it up with your own
  evidence, or someone else's.
\end{enumerate}

Compressing all of that into 8000 words is much more challenging than
filling 40 pages with unnecessary text and graphics. It also requires
much more cooperation on and discussion of what needs to go into those
pages to make them as effective as possible.

A good group member is one that can read a piece of writing critically,
and give constructive feedback -- to do this effectively is necessary
for all group members to be clear about the ``red thread'' in the
project (the ``why'').

I would personally prefer that you write about 15 pages of really good
work, than 40 pages of low quality work.

\textbf{For a journal article size paper these are some rough
guidelines:}

The share of pages between the sections depends on how much space
you\emph{need}. I say\emph{need}, because people reading your work want
to get the clearest message, in as few words as possible. A (very) rough
guide as to how many (academic) references each section could have is
included in red text.

\begin{enumerate}
\def\labelenumi{\arabic{enumi}.}
\item
  Abstract (+-150 words)
\item
  Introduction (0.75 -- 1.25 pages) (Motivation, justification,
  explanation of why?(4 -- 5 references))
\item
  Literature / theory (1.5 -- 3 pages, depending on how theoretical your
  paper is) (Demonstrate reading --(6 -- 12 references))
\item
  Method (0.75 -- 3 pages, depending on how complex the explanation
  needs to be) (Justify choice, explain details(4 -- 5 references))
\item
  Results (1 -- 3 pages) (Presentation of results (2 -- 5 references))
\item
  Discussion (2 -- 5 pages) (Interpretations, comparisons,
  perspectives(4 -- 5 references))
\item
  Conclusion (1 page) (Link discussion to introduction(No new
  references))
\end{enumerate}

(The max pages in this example is 16.5 pages -- it is just an example,
and the split between the sections will change depending on the type of
research.)

(Min references in this example is 20, but this is on the high side. You
won't have time to read as much as that. 8 -- 15 references in total
should be enough if you find some really good ones.)

Keep it simple! That is the best advice I ever got\ldots{} and the
hardest to follow, because you really need to be sure of what you're
talking about to write clearly and simply.

\hypertarget{rough-guide-to-writing-a-problem-statement}{%
\subsection{Rough guide to writing a problem
statement}\label{rough-guide-to-writing-a-problem-statement}}

Coming up with a good question does not mean coming up with a question
that will change the world. It means coming up with a question that you
can answer \textbf{in the time you are allowed}, and\textbf{with the
tools you have (or have time to learn)}.

A note on how to get started with your problem selection -- try to be
curious. It is going to take some active effort.

\textbf{Most of you have identified an area of interest, rather than any
specific question.} These are just a few pointers to help you to
identify a good problem.

\begin{itemize}
\tightlist
\item
  If you want to read a pretty good ``how-to'' guide, try this
  one:\href{https://www.wikihow.com/Write-a-Problem-Statement}{{https://www.wikihow.com/Write-a-Problem-Statement}}
\end{itemize}

You need to actively apply yourselves to \emph{finding} a question:

\hypertarget{step-1-brainstorm}{%
\subsubsection{Step 1: Brainstorm:}\label{step-1-brainstorm}}

\begin{itemize}
\item
  Find a meeting room and mind-map an area of economics that you are
  interested in.
\item
  A very high level overview of the courses (like the table of contents
  in your text books) you have done so far should help you to understand
  the tools that you have, which you can use to answer whatever question
  you end up asking.

  \begin{itemize}
  \tightlist
  \item
    Some of these tools will be theories, models, data types and
    sources. The learning outcomes of your courses are also a good guide
    (check Moodle).
  \end{itemize}
\item
  If you really want to be active in your search for problems, and deal
  with real life issues -- pick up a phone and make some calls to people
  in the area or industry you want to look at. These kinds of discussion
  can be really motivating and insightful.
\end{itemize}

As a group you can really benefit by getting ideas and input from each
other. This does not need to take a long time but does require effort.

Don't get caught up by not deciding what to do -- get into a room, set a
time frame and get finished with the choice early.

\hypertarget{step-2-getting-that-problem-statement-clear}{%
\subsubsection{Step 2: Getting that problem statement
clear}\label{step-2-getting-that-problem-statement-clear}}

Where to start?

Read.

\hypertarget{using-the-the-introduction-as-the-route-to-the-problem-statement}{%
\subsubsection{Using the the introduction as the route to the problem
statement}\label{using-the-the-introduction-as-the-route-to-the-problem-statement}}

A good introduction leads to a good problem statement, but this is only
possible if you have some good material to work with.

\hypertarget{example-flow-of-a-good-introduction}{%
\paragraph{Example flow of a good
introduction:}\label{example-flow-of-a-good-introduction}}

\begin{enumerate}
\def\labelenumi{\arabic{enumi}.}
\tightlist
\item
  You could list some shocking figures or numbers that highlight that
  there is some area / issue that we should be concerned about. It could
  also be some clear contradiction or controversy in the literature that
  needs clarification - but this requires some careful reading.
\end{enumerate}

\begin{quote}
(This defines the problem area that you want to address - normally these
are the effects of some other problem that can be identified.)
\end{quote}

\begin{enumerate}
\def\labelenumi{\arabic{enumi}.}
\setcounter{enumi}{1}
\tightlist
\item
  You could then explain that some people see (explain) these
  facts/figures from the perspective of A / B / C \ldots{} and conclude
  that X / Y / Z. Another group of researchers suggest that there may be
  another way to interpret these figures.. they say\ldots{} etc.
\end{enumerate}

\begin{quote}
(This highlights existing theories in the literature - these could be
conflicting explanations for the effects that are observed by different
researchers. Always remember that research can have a particular agenda,
i.e.~to promote efficiency of markets, or to critique main-stream
literature - so be a little critical when reading. Every researcher is
trying to sell their ideas.)
\end{quote}

\begin{enumerate}
\def\labelenumi{\arabic{enumi}.}
\setcounter{enumi}{2}
\tightlist
\item
  Next, you could explain that `these' different approaches suggest that
  this problem could be investigated by doing E/F/G.
\end{enumerate}

\begin{quote}
(Identification of possible ways of addressing the issue - There are a
variety of quantitative and qualitative methods, and some people rely
more on some than on others. This section should still be able to refer
to how other people have addressed the issue. Here is where you narrow
down what might be a good way to look at the problem.)
\end{quote}

\begin{enumerate}
\def\labelenumi{\arabic{enumi}.}
\setcounter{enumi}{3}
\tightlist
\item
  More specifically, therefore, we want to investigate how / if / what
  the effects of / etc.
\end{enumerate}

\begin{quote}
(By this stage you should be able to be specific about the exact part of
the problem that you are interested in addressing, and all of the
arguments for why it is interesting should already have been made.)
\end{quote}

\hypertarget{some-general-notes-about-writing-a-problem-statement}{%
\subsubsection{Some general notes about writing a problem
statement}\label{some-general-notes-about-writing-a-problem-statement}}

The general structure of questions in most economic analyses is:

\begin{quote}
What is, has been, or will be the impact of\textbf{A}on\textbf{B}?
\end{quote}

\hypertarget{where-do-you-find-a-problem-in-macroeconomics}{%
\paragraph{\texorpdfstring{\textbf{Where do you find a problem in
macroeconomics?}}{Where do you find a problem in macroeconomics?}}\label{where-do-you-find-a-problem-in-macroeconomics}}

Many economic queries attempt to measure social consequences, and most
often try to assign a financial value to those consequences. Sometimes
this is phrased as a ``comparative outcome'' or ``alternative
scenario''. Normally this means that one option is \emph{better} than
another. This is not surprising, since pretty much all macroeconomic
variables are aggregated financial values or derivatives thereof.

\begin{verbatim}
Other ways to frame a question could be:
\end{verbatim}

\begin{enumerate}
\def\labelenumi{\arabic{enumi}.}
\item
  Why is one option better than the other? How can you tell?
\item
  How does a change in X affect income for the economy / households /
  teachers?
\end{enumerate}

\begin{itemize}
\item
  How many people are employed?
\item
  Labour productivity?
\item
  How is the standard of living affected?
\end{itemize}

These are justification options (evidence) to motivate \emph{why} an
investigation is a good idea. Just remember that you need to be able to
find the evidence to back up your arguments. As noted in the example of
an introduction flow above, your selection of a method should be the
logical conclusion of your reading.

\hypertarget{answering-the-why}{%
\subsubsection{\texorpdfstring{Answering the
\emph{Why?}}{Answering the Why?}}\label{answering-the-why}}

Why is it that investigating this problem will be beneficial? To answer
the question generally requires awareness of:

\begin{enumerate}
\def\labelenumi{\arabic{enumi}.}
\tightlist
\item
  What?
\item
  When?
\item
  Where?
\item
  Who?
\item
  In what way?
\item
  How?
\end{enumerate}

These things are the specifics of the problem statement, but they will
not provide you with the answer to why it is important.

The best place to find an answer to \emph{why?} is to read some of the
most recent work on the topic you have found. Anything that has been
published has been read and edited by at least 5 people by the time it
goes to a journal. The arguments in those articles should therefore be
pretty reasonable.

Once you understand your \emph{why?} you can start with the specifics

\hypertarget{what}{%
\paragraph{What?}\label{what}}

For example: If you chose to research currency markets some of the
sub-categories could be (What?):

\begin{verbatim}
-   Trading platforms

-   Exchange rate policies

-   Regulations

-   Common currency areas

-   Clearing and Settlement systems

-   Speculation

-   Risk-mitigation / hedging
\end{verbatim}

This delimitation is often determined by the problem area that you
identify. Normally illustrated by something interesting or concerning,
which leads you to think that investigating the area might be
interesting.

If you move into academic writing (i.e.~for journals) you can go the
other way around, and check what has been written about recently or is
currently being funded at the EU or national levels. Read some of the
latest literature relating to it and then identify if there is a similar
problem in an area that has not yet been investigated.

\hypertarget{when}{%
\paragraph{When?}\label{when}}

You need to be carefully aware of what time period you choose to
investigate, as it has direct consequences for the types of conclusions
you can make from your analysis. If you look at 1950 -1970 consumption
data, you can't really say anything intelligent about the use of
disposable income in 2020.

Historical analysis is crucial in economics, but you should always be
aware of how the structure of the economies of the world have shifted
over time.

Education rates, the type of institutions that exist, levels of
unemployment, the size of government, etc.

A trendy catch phrase for this kind of context is PESTLE, an anagram for
how the context of countries change: 1. Political 2. Economic 3. Social
4. Technological 5. Legal 6. Environmental

Please don't write this list out and do a PESTLE analysis in your
projects, it is just to tell you that there are many changes that take
place over time. And that you need to think specifically about time in
your problem statement.

\hypertarget{where}{%
\paragraph{Where?}\label{where}}

This is quite obvious, but is not only related to geographical borders.
For example, a study could be:

\begin{verbatim}
-   National

-   Municipal

-   Regional

-   Global
\end{verbatim}

\hypertarget{who}{%
\paragraph{Who?}\label{who}}

Which groups are involved in your project, and who is it that will be
interested in reading the results of your research? Who will the
research / problem investigation be useful or interesting for.

\begin{verbatim}
-   Demographic specific (Ex., students, low-income families, employees at public institutions, etc)

-   Institution specific

-   Industry specific

-   Country specific
\end{verbatim}

\hypertarget{in-what-way}{%
\paragraph{In what way?}\label{in-what-way}}

You also need to know what kind of impact or relationship you are
looking for. Is it, \emph{how much of the behaviour of A can be
explained by B}? Is it a \emph{theoretical or empirical issue} or are
you interested in \emph{How much influence will A have on B? Or vice
versa, or both}?

\hypertarget{how}{%
\paragraph{How?}\label{how}}

Finally you need to be able to explain how you are going to investigate
the problem. You should consider your tools and your course on
methodology (and theory of science) to be able to figure out how will
you answer the question.

Recall that methodology is layered: \textgreater{}Philosophy of science
-\textgreater{} Methodology -\textgreater{} Theory -\textgreater{}
Methods

Methods are at the bottom of the pyramid, and can be quantitative or
qualitative, or a combination of both. It is important to be aware of
what you are doing, and using. They are normally used to motivate one or
another theory.

The philosophy of science defines what kind of results will be
considered valid. For example, is it valid to make a conclusion about
the future based on the past (i.e.~to use data to make predictions)?

Some examples of this are: 1. Cartesian / Euclidian theories of science
2. The Babylonian mode of thinking

Theories use methods, methods are part of a particular methodology, and
the methodology is made valid by the theory of science inside which it
fits.

\hypertarget{stay-aware-of-your-time-limitations}{%
\subsection{Stay aware of your time
limitations}\label{stay-aware-of-your-time-limitations}}

Be aware of the time you have left to do the research. Data /
information collection and organisation takes time, and you need to get
started on it early if you want to be able to say anything useful by the
time you finish writing your project up.

A brief guide:

\begin{enumerate}
\def\labelenumi{\arabic{enumi}.}
\item
  Problem identification and reading: 1 week
\item
  Write literature review: 1 week
\item
  Write first draft of introduction: 2 days
\item
  Data collection: 1 week
\item
  Write method and methodology section: 3 days
\item
  Data cleaning and analysis: 2 weeks
\item
  Write results: 1 week
\item
  Write conclusion: 2 days
\item
  Re-write introduction: 2 days
\item
  Check document for references and errors: 1 day
\end{enumerate}

\FloatBarrier
\newpage

\fancyhead[CO,CE]{Part 1 header for page - change this, \arabic{page}}

\hypertarget{first-section-not-numbered-because-of-the-curly-bracket-dash--}{%
\subsection*{First section, not numbered because of the curly bracket
dash
\{-\}}\label{first-section-not-numbered-because-of-the-curly-bracket-dash--}}
\addcontentsline{toc}{subsection}{First section, not numbered because of
the curly bracket dash \{-\}}

Curabitur pretium tincidunt lacus. Nulla gravida orci a odio. Nullam
varius, turpis et commodo pharetra, est eros bibendum elit, nec luctus
magna felis sollicitudin mauris. Brown \& Graf (2013) noted that Integer
in mauris eu nibh euismod gravida. Duis ac tellus et risus vulputate
vehicula. Donec lobortis risus a elit. Etiam Lorem ipsum dolor sit amet,
consectetur adipiscing elit, sed do eiusmod tempor incididunt ut labore
et dolore magna aliqua. Ut enim ad minim veniam, quis nostrud
exercitation ullamco laboris nisi ut aliquip ex ea commodo consequat.
Duis aute irure dolor in reprehenderit in voluptate velit esse cillum
dolore eu fugiat nulla pariatur. Excepteur sint occaecat cupidatat non
proident, sunt in culpa qui officia deserunt mollit anim id est laborum.

Andersen, Duus, \& Jensen (2016, p. 25)

Justiniano, Primiceri, \& Tambalotti (2015, pp. 10--25)

(See, Zinman, 2015, p. 15)

Curabitur pretium tincidunt lacus. Nulla gravida orci a odio. Nullam
varius, turpis et commodo pharetra, est eros bibendum elit, nec luctus
magna felis sollicitudin mauris. Integer in mauris, as suggested by
Walks (2013, pp. 10--25), eu nibh euismod gravida. Duis ac tellus et
risus vulputate vehicula. Donec lobortis risus a elit. Etiam tempor. Ut
ullamcorper, ligula eu tempor congue, eros est euismod turpis, id
tincidunt sapien risus a quam. Maecenas fermentum consequat mi. Donec
fermentum. Pellentesque malesuada nulla a mi. Duis sapien sem, aliquet
nec, commodo eget, consequat quis, neque. Aliquam faucibus, elit ut
dictum aliquet, felis nisl adipiscing sapien, sed malesuada diam lacus
eget erat. Cras mollis scelerisque nunc. Nullam arcu. Aliquam consequat.
Curabitur augue lorem, dapibus quis, laoreet et, pretium ac, nisi.
Aenean magna nisl, mollis quis, molestie eu, feugiat in, orci. In hac
habitasse platea dictumst.

\hypertarget{sec:part1-sec2}{%
\subsection{Second section numbered}\label{sec:part1-sec2}}

Curabitur pretium tincidunt lacus. Nulla gravida orci a odio. Nullam
varius, turpis et commodo pharetra, est eros bibendum elit, nec luctus
magna felis sollicitudin mauris. Integer in mauris eu nibh euismod
gravida. (See, Zinman, 2015, p. 15) Duis ac tellus et risus vulputate
vehicula. Donec lobortis risus a elit. Etiam Lorem ipsum dolor sit amet,
consectetur adipiscing elit, sed do eiusmod tempor incididunt ut labore
et dolore magna aliqua. Ut enim ad minim veniam, quis nostrud
exercitation ullamco laboris nisi ut aliquip ex ea commodo consequat.
Duis aute irure dolor in reprehenderit in voluptate velit esse cillum
dolore eu fugiat nulla pariatur. Excepteur sint occaecat cupidatat non
proident, sunt in culpa qui officia deserunt mollit anim id est laborum.

\FloatBarrier
\cleardoublepage

\fancyhead[CO,CE]{Part 2 title page header for page - change this}

\begin{centering}

\vspace{1 cm}

\Huge

\section*{Header for section}
\addcontentsline{toc}{section}{Header you want in the TOC for part 1}
\addtocounter{section}{1}

\vspace{1 cm}

\Large
Name 1

\normalsize
Supervisor: Supervisor name

\vspace{1 cm}

\Large


\vspace{1 cm}

\normalsize
The Department of Business and Management

Aalborg University

\vspace{1 cm}

\end{centering}

\FloatBarrier
\newpage

\fancyhead[CO,CE]{Part 2 header with a page number, \arabic{page}}

\hypertarget{sec:part2-sec1}{%
\subsection{First section}\label{sec:part2-sec1}}

Curabitur pretium tincidunt lacus. Nulla gravida orci a odio. Nullam
varius, turpis et commodo pharetra, est eros bibendum elit, nec luctus
magna felis sollicitudin mauris. Integer in mauris eu nibh euismod
gravida. Duis ac tellus et risus vulputate vehicula. Donec lobortis
risus a elit. Etiam Lorem ipsum dolor sit amet, consectetur adipiscing
elit, sed do eiusmod tempor incididunt ut labore et dolore magna aliqua.
Ut enim ad minim veniam, quis nostrud exercitation ullamco laboris nisi
ut aliquip ex ea commodo consequat. Duis aute irure dolor in
reprehenderit in voluptate velit esse cillum dolore eu fugiat nulla
pariatur. Excepteur sint occaecat cupidatat non proident, sunt in culpa
qui officia deserunt mollit anim id est laborum.

\emph{Some random italics text}

\textbf{Some random bold text}

A pretty cool reference back to the first section can be made
automatically like this: \ref{sec:part1-sec2}

The standard command will just generate a number, like this :
\ref{sec:part1-sec2}

You can also use the command created in the ``ProjectBuilder'' to
customise your references.

This command includes the word ``Section'' before the reference:
Section~\ref{sec:part1-sec2}

You can do the same for figures and tables. If you label them correctly,
then they will automatically be added to the list of tables and figures
at the end of the document.

Curabitur pretium tincidunt lacus. \emph{Some random italics text}, est
eros bibendum elit, nec luctus magna felis sollicitudin mauris. Integer
in mauris eu nibh euismod gravida. Duis ac tellus et risus vulputate
vehicula. Donec lobortis risus a elit. Etiam tempor. Ut ullamcorper,
ligula eu tempor congue, eros est euismod turpis, id tincidunt sapien
risus a quam. Maecenas fermentum consequat mi. Donec fermentum.
Pellentesque malesuada nulla a mi. Duis sapien sem, aliquet nec, commodo
eget, consequat quis, neque. Aliquam faucibus, elit ut dictum aliquet,
felis nisl adipiscing sapien, sed malesuada diam lacus eget erat. Cras
mollis scelerisque nunc. Nullam arcu. Aliquam consequat. Curabitur augue
lorem, dapibus quis, laoreet et, pretium ac, nisi. Aenean magna nisl,
mollis quis, molestie eu, feugiat in, orci. In hac habitasse platea
dictumst.

\hypertarget{sec:part2-sec2}{%
\subsection{Second section numbered}\label{sec:part2-sec2}}

Curabitur pretium tincidunt lacus. Nulla gravida orci a odio. Nullam
varius, turpis et commodo pharetra, est eros bibendum elit, nec luctus
magna felis sollicitudin mauris. Integer in mauris eu nibh euismod
gravida. Duis ac tellus et risus vulputate vehicula. Donec lobortis
risus a elit. Etiam Lorem ipsum dolor sit amet, consectetur adipiscing
elit, sed do eiusmod tempor incididunt ut labore et dolore magna aliqua.
Ut enim ad minim veniam, quis nostrud exercitation ullamco laboris nisi
ut aliquip ex ea commodo consequat. Duis aute irure dolor in
reprehenderit in voluptate velit esse cillum dolore eu fugiat nulla
pariatur. Excepteur sint occaecat cupidatat non proident, sunt in culpa
qui officia deserunt mollit anim id est laborum.

\hypertarget{sec:part2-sec2-subsec1}{%
\subsubsection{A third level heading}\label{sec:part2-sec2-subsec1}}

Integer in mauris eu nibh euismod gravida. Duis ac tellus et risus
vulputate vehicula. Donec lobortis risus a elit. Etiam Lorem ipsum dolor
sit amet, consectetur adipiscing elit, sed do eiusmod tempor incididunt
ut labore et dolore magna aliqua. Ut enim ad minim veniam, quis nostrud
exercitation ullamco laboris nisi ut aliquip ex ea commodo consequat.
Duis aute irure dolor in reprehenderit in voluptate velit esse cillum
dolore eu fugiat nulla pariatur. Excepteur sint occaecat cupidatat non
proident, sunt in culpa qui officia deserunt mollit anim id est laborum.

\begin{quote}
"quoted text of some fancy academic somewhere\ldots{}Duis ac tellus et
risus vulputate vehicula. Donec lobortis risus a elit. Etiam tempor. Ut
ullamcorper, ligula eu tempor congue, eros est euismod turpis, id
tincidunt sapien risus a quam. Maecenas fermentum consequat mi. Donec
fermentum. Pellentesque malesuada nulla a mi. Duis sapien sem, aliquet
nec, commodo eget, consequat quis, neque. Aliquam faucibus, elit ut
dictum aliquet, felis nisl adipiscing sapien, sed malesuada diam lacus
eget erat. Cras mollis scelerisque nunc. Nullam arcu. Aliquam consequat.
Curabitur augue lorem, dapibus quis, laoreet et, pretium ac, nisi.
Aenean magna nisl, mollis quis, molestie eu, feugiat in, orci. In hac
habitasse platea dictumst.
\end{quote}

\hypertarget{sec:part2-sec2-subsec2}{%
\subsubsection{A second level 3 heading}\label{sec:part2-sec2-subsec2}}

This section contains some fancy equations these can be entered in a
couple of ways, but the LaTeX version is probably the most adaptable and
consitent. in text you can just type something in single \$ marks, and
it will stay in line,
\(y = \beta_0 + \beta _1 ln(x^2) + \frac{ac}{x^2_1}\).

Or you can type it in double \$\$ marks and it will float the equation
in the middle of the page with white space around it.

\[y = \beta_0 + \beta _1 ln(x^2) + \frac{ac}{x^2_1}\]

If you want to get really fancy and to be able to reference your
equations automatically, you can also add a label, but this is easier to
do in the LaTeX version.

\begin{equation}
y = \beta_0 + \beta _1 ln(x^2) + \frac{ac}{x^2_1}
\label{eq:fancy-equation}
\end{equation}

This version allows you to do a reference to Equation
(\ref{eq:fancy-equation}) in your text. If you look at the code, you
will see that there are no section numbers or equation numbers ``hard
coded'' into the text. If I add another equation above this one, it will
just adust the numbering automatically.

The strucutre of the numbering of different elements is set in the
\texttt{YAML} of the ``ProjectBuilder''. There you will find code that
looks like this
\texttt{\textbackslash{}counterwithin\{table\}\{section\}} - this means,
that every time a level 1 section is started, the counter for tables
will start from 1 again.

\hypertarget{for-inserting-images}{%
\subsection{For inserting images}\label{for-inserting-images}}

you might also want to do some R analysis in your document and then
output the results in the text!

\begin{center}\includegraphics{figures/unnamed-chunk-8-1} \end{center}

\FloatBarrier
\cleardoublepage

\pagenumbering{roman}

\fancyhead[LO,RE]{}
\fancyhead[CO,CE]{List of Figures, \arabic{page}}

\listoffigures
\addcontentsline{toc}{section}{List of Figures}

\FloatBarrier
\cleardoublepage

\fancyhead[CO,CE]{List of Tables, \arabic{page}}

\listoftables
\addcontentsline{toc}{section}{List of Tables}

\newpage
\FloatBarrier

\cleardoublepage
\fancyhead[CO,CE]{References, \arabic{page}}

\hypertarget{references}{%
\section*{References}\label{references}}
\addcontentsline{toc}{section}{References}

\hypertarget{refs}{}
\leavevmode\hypertarget{ref-Andersen2016}{}%
Andersen, A. L., Duus, C., \& Jensen, T. L. (2016). Household debt and
spending during the financial crisis: Evidence from Danish micro data.
\emph{European Economic Review}, \emph{89}, 96--115. Retrieved from
\url{http://linkinghub.elsevier.com/retrieve/pii/S0014292116301106}

\leavevmode\hypertarget{ref-Brown2013}{}%
Brown, M., \& Graf, R. (2013). Financial Literacy, Household Investment
and Household Debt: Evidence from Switzerland. \emph{Working Papers on
Finance}. Retrieved from
\url{http://www1.vwa.unisg.ch/RePEc/usg/sfwpfi/WPF-1301.pdf}

\leavevmode\hypertarget{ref-Justiniano2015b}{}%
Justiniano, A., Primiceri, G., \& Tambalotti, A. (2015). Household
leveraging and deleveraging. \emph{Review of Economic Dynamics},
\emph{18}(1), 3--20.

\leavevmode\hypertarget{ref-Walks2013}{}%
Walks, A. (2013). Mapping the Urban Debtscape: The Geography of
Household Debt in Canadian Cities. \emph{Urban Geography}, \emph{34}(2),
153--187. Routledge.

\leavevmode\hypertarget{ref-Zinman2015}{}%
Zinman, J. (2015). Household Debt: Facts, Puzzles, Theories, and
Policies. \emph{Annual Review of Economics}, \emph{7}(1), 251--276.
Annual Reviews. Retrieved from
\url{http://www.annualreviews.org/doi/10.1146/annurev-economics-080614-115640}


\end{document}
